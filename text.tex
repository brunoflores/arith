\documentclass{article}
\usepackage[utf8]{inputenc}
\usepackage{mathtools}
\usepackage{soul}
\usepackage{float}

\title{Types and Programming Languages (Pierce)}
\author{Bruno Flores}
\date{September 2021}

\begin{document}

\maketitle

\section{Untyped Arithmetic Expressions}

\subsection{Syntax}

\begin{figure}[h!]
    \begin{verbatim}
        t ::=
            true
            false
            if t then t else t
            0
            succ t
            pred t
            iszero t
    \end{verbatim}
    \caption{BNF}
\end{figure}

\begin{figure}[h!]
    \begin{enumerate}
        \item \{\texttt{true, false, 0}\} \(\subseteq\) \(\mathcal{T}\);
        \item 
            if \texttt{t\(_1\)} \(\subseteq\) \(\mathcal{T}\), 
            then \{\texttt{succ t\(_1\), pred t\(_1\), iszero t\(_1\)}\} \(\subseteq\) \(\mathcal{T}\);
        \item 
            if \texttt{t\(_1\)} \(\subseteq\) \(\mathcal{T}\), \texttt{t\(_2\)} \(\subseteq\) \(\mathcal{T}\),
            and \texttt{t\(_3\)} \(\subseteq\) \(\mathcal{T}\), 
            then \texttt{if t\(_1\) then t\(_2\) else t\(_3\)} \(\subseteq\) \(\mathcal{T}\).
    \end{enumerate}
    \caption{Terms, inductively}
\end{figure}

\begin{table}[h!]
    \centering
    \begin{tabular}{c c c}
         \texttt{true} \(\subseteq\) \(\mathcal{T}\) & \texttt{false} \(\subseteq\) \(\mathcal{T}\) & \texttt{0} \(\subseteq\) \(\mathcal{T}\) \\
         \\
         \begin{tabular}{c}
              t\(_1\) \(\subseteq\) \(\mathcal{T}\) \\
              \hline
              \texttt{succ t\(_1\)} \(\subseteq\) \(\mathcal{T}\)
         \end{tabular} & \begin{tabular}{c}
              t\(_1\) \(\subseteq\) \(\mathcal{T}\) \\
              \hline
              \texttt{pred t\(_1\)} \(\subseteq\) \(\mathcal{T}\)
         \end{tabular} & \begin{tabular}{c}
              t\(_1\) \(\subseteq\) \(\mathcal{T}\) \\
              \hline
              \texttt{iszero t\(_1\)} \(\subseteq\) \(\mathcal{T}\)
         \end{tabular} \\
         \\
         \multicolumn{3}{c}{
             \begin{tabular}{c}
                  t\(_1\) \(\subseteq\) \(\mathcal{T}\) t\(_2\) \(\subseteq\) \(\mathcal{T}\) t\(_3\) \(\subseteq\) \(\mathcal{T}\) \\
                  \hline
                  \texttt{if t\(_1\) then t\(_2\) else t\(_3\)} \(\subseteq\) \(\mathcal{T}\)
             \end{tabular}
         }
    \end{tabular}
    \caption{Terms, by inference rules}
    \label{tab:my_label}
\end{table}

\begin{table}[h!]
    \centering
    \begin{tabular}{l l l l}
        S\(_0\) & = & \(\emptyset\) & \\
        S\(_{\textit{i}+1}\) & = & & \{\texttt{true, false, 0}\} \\
        & & \(\cup\) & \{\texttt{succ t\(_1\), pred t\(_1\), iszero t\(_1\)} \(|\) \texttt{t\(_1\)} \(\in\) S\(_i\)\} \\
        & & \(\cup\) & \{\texttt{if t\(_1\) then t\(_2\) else t\(_3\)} \(|\) \texttt{t\(_1\), t\(_2\), t\(_3\)} \(\in\) S\(_i\)\}
    \end{tabular}
    \caption{Terms, concretely}
\end{table}

Let

\begin{center}
    S = \(\underset{i}{\text{\Large\(\cup\)}}\) S\(_i\).
\end{center}

S\(_0\) is empty; S\(_1\) contains just the constants; S\(_2\) contains the constants plus the phrases that can be built with constants and just one \texttt{succ, pred, iszero}, or \texttt{if}; S\(_3\) contains these and all phrases that can be built using \texttt{succ, pred, iszero}, and \texttt{if} on phrases in S\(_2\); and so on. \hl{S collects together all the phrases that can be built this way.}

\end{document}